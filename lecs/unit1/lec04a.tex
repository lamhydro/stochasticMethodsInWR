\documentclass[8pt]{beamer}
\usetheme{default} % Very simple theme
\usecolortheme{dove} % Minimalist grayscale colors
\setbeamertemplate{navigation symbols}{} % Remove navigation bar

% Define brand-like colors
\definecolor{myblue}{RGB}{0, 102, 204}    % Blue for standard blocks
\definecolor{mygreen}{RGB}{0, 153, 0}     % Green for example blocks
\definecolor{myred}{RGB}{204, 0, 0}       % Red for alert blocks
\definecolor{myorange}{RGB}{255, 128, 0} % For \emph

% Accent color for titles, bullets, etc.
\setbeamercolor{structure}{fg=myblue}

% Block colors
\setbeamercolor{block title}{bg=myblue, fg=white}
\setbeamercolor{block body}{bg=blue!5, fg=black}

\setbeamercolor{exampleblock title}{bg=mygreen, fg=white}
\setbeamercolor{exampleblock body}{bg=green!5, fg=black}

\setbeamercolor{alertblock title}{bg=myred, fg=white}
\setbeamercolor{alertblock body}{bg=red!5, fg=black}
\setbeamercolor{alerted text}{fg=myred}

% Optional: color for \example text
\setbeamercolor{example text}{fg=mygreen}

% Custom emph (orange text instead of italic)
\let\oldemph\emph
\renewcommand{\emph}[1]{\textcolor{myorange}{#1}}


\usepackage{graphicx} % Package for images
\usepackage{amsmath} % Package for images
\graphicspath{{figs/}} % Path to your figures directory
\usepackage{array}  % For better column spacing
\usepackage{caption} % Required for \captionof

% Title Information
\title{STOCHASTIC METHODS IN WATER RESOURCES}
\vspace{25pt}
\subtitle{Unit 1: Introduction to probability and statistics \\ Lecture 4a: Model estimation and testing}
\author{Luis Alejandro Morales, Ph.D.}
\institute{Universidad Nacional de Colombia \\ Department of Civil and Agriculture Engineering} %// }
%\date{\today}

\begin{document}

% Title Slide
\begin{frame}
    \titlepage
\end{frame}

%-------
% From Kotte
\section{Generalities}
\begin{frame}{Generalities}
    \begin{itemize}
        \item Statistical inference deals with  statistical estimations based on a \emph{sample} from the \emph{population}.
        \item Some definitions:
            \begin{itemize}
                \item \alert{Population}: consist of all possible observations of a process (e.g. air temperature at certain location). Some of the observations in the in the population may not have any physical sense, perhabs, due to sensor errors. 
                \item \alert{Sample}: is a subset of the population (e.g. instantaneous daily streamflow for a certain period in a station). A \emph{random sample} is thus a sample that is representative of the population.
                \item \alert{Random variables}: is a \emph{real-valued function} defined on a \emph{sample space}. Wheather a random variable is \emph{discrete} or \emph{continuous} depends on how the sample space is defined. 
                \item \alert{Statistic}: is a function of the observations that is quantifiable and does not contain any unknown parameter. Note that a \emph{statistic} is also a random variable that provides an \emph{estimation}.
                \item \alert{Estimator}: is the method or rule of estimation. For instance, the \emph{sample mean} $\bar{X}$ is a point estimator of the \emph{population mean} $\mu$. 
                \item \alert{Estimate}: is the value yielded by the estimator. 
                 \end{itemize}
             \item Suppose that the population of variable $X$ follows a \emph{Normal distribution} and the distribution parameters $\theta$ are unknown. Thus, a random sample of $X$ of size $n$.
             \item Parameters $\theta$ can be described by a number or a range; this last include an uncertainty. 
    \end{itemize}
\end{frame}

%-------
% From Kotte
\section{Properties of estimators}
\begin{frame}{Properties of estimators}

\end{frame}
    




\end{document}
